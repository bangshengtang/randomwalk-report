\subsubsection{Definitions}
In contrast to the undirected cases, the transition matrix would not be symmetric
here.

Analogues of Cheeger's constant could also be defined, and similar inequalities
hold.

\subsubsection{Perron-Fr\"obenius Theorem}
This theorem is a basic result in matrix analysis.

\begin{theorem}
Let $M$ be a matrix with nonnegative entries. Assume furthermore that $M$ is
\emph{irreducible} meaning that it is impossible to permute the rows and columns of
$M$ to place it in the form,
\begin{eqnarray*}
M'=\left(
\begin{array}{cc}
A & 0\\
B & C
\end{array}
\right)
\end{eqnarray*}
with the upper right block having dimension $k\times (n-k)$ for some $k$. Then there
is a real $\rho_0>0$ such that the following hold:
\begin{enumerate}[1.]
\item $\rho_0$ is an eigenvalue of $M$, and all other eigenvalues satisfy $|\rho|\le
\rho_0$.
\item The eigenvector corresponding to the eigenvalue $\rho_0$ has nonnegative
entries.
\item If there are $k-1$ other eigenvalues with $|\rho_i|=\rho_0$, then they are of
the form $\rho_0\theta^j$, where $\theta=e^{2\pi i\over k}.$
\end{enumerate}

\end{theorem}

When translating this theorem into graph theoretic terms and using the facts about
the transition matrix of a directed graph, the following theorems hold,
\begin{theorem}
A strongly connected directed graph $G$ has a maximum eigenvalue $\rho_0$, and that
eigenvalue has a corresponding non-negative eigenvector. Furthermore,
$|\rho_i|<\rho_0$ for all $i>1$ iff the GCD of the cycle lengths of $G$ is 1.
\end{theorem}

\begin{theorem}
The random walk on a directed graph $G$ converges to a unique stationary
distribution if $G$ is strongly connected and is not periodic.
\end{theorem}
