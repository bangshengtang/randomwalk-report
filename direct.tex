For simplicity, the previous sections are devoted mainly to
undirected graphs, one might figure out that directed cases are also
of interest. However, as mentioned in previous sections, there are
some properties hold only in undirected cases. In this section, some
analogues to undirected cases are given, and the issue of
convergence is also discussed with the help of a basic matrix
analytical tool.

\subsection{Definitions}
The transition matrix is defined exactly the same way with the
undirected case, but in contrast to the latter, the transition
matrix is generally not symmetric.

Analogues of Cheeger's constant could also be defined, and similar inequalities
hold.

\subsection{Perron-Fr\"obenius Theorem}
This theorem is a basic result in matrix analysis.

\begin{theorem}
Let $M$ be a matrix with nonnegative entries. Assume furthermore that $M$ is
\emph{irreducible} meaning that it is impossible to permute the rows and columns of
$M$ to place it in the form,
\begin{eqnarray*}
M'=\left(
\begin{array}{cc}
A & 0\\
B & C
\end{array}
\right)
\end{eqnarray*}
with the upper right block having dimension $k\times (n-k)$ for some $k$. Then there
is a real $\rho_0>0$ such that the following hold:
\begin{enumerate}%[1.]
\item $\rho_0$ is an eigenvalue of $M$, and all other eigenvalues satisfy $|\rho|\le
\rho_0$.
\item The eigenvector corresponding to the eigenvalue $\rho_0$ has nonnegative
entries.
\item If there are $k-1$ other eigenvalues with $|\rho_i|=\rho_0$, then they are of
the form $\rho_0\theta^j$, where $\theta=e^{2\pi i\over k}.$
\end{enumerate}

\end{theorem}

When translating this theorem into graph theoretic terms and using the facts about
the transition matrix of a directed graph, the following theorems hold,
\begin{theorem}
A strongly connected directed graph $G$ has a maximum eigenvalue $\rho_0$, and that
eigenvalue has a corresponding non-negative eigenvector. Furthermore,
$|\rho_i|<\rho_0$ for all $i>1$ iff the GCD of the cycle lengths of $G$ is 1.
\end{theorem}

\subsection{Convergence on Directed Graphs}
As a matter of fact, by the Perron-Fr\"obenius theorem, the
condition that $G$ is strongly connected and the GCD of the cycle
lengths in $G$ is $1$ is equivalent to the statement that the norms
of all eigenvalues of $P$ except for $\rho_0=1$ are less than $1$.
Thus one may derive the convergence of a random walk from a spectral
argument.

With the condition stated above, consider probability transition
matrix $P$, take $P_0=\overrightarrow{1}\phi$, where $\phi$ is the
row eigenvector of $P$ associated with eigenvalue $\rho_0=1$ with
positive entries which sum to $1$, and let $M=P-P_0$. The following
three claims hold.

\begin{claim}
$P-P_0$ has the same spectral with $P$ except that in $P-P_0$, the
eigenvalue $1$ is replaced by $0$.
\end{claim}
\begin{proof}
$P_0$ has an eigenvalue of $1$ and $n-1$ multiple eigenvalues of
$0$. Obviously, $\phi$ is an eigenvector of $P_0$. Thus,
\begin{eqnarray*}
\rho\cdot (P-P_0)=\rho\cdot P-\rho\cdot P_0=0
\end{eqnarray*}

For another eigenvector $v$ for $P$, since it corresponds to a
different eigenvalue from $1$, it is orthogonal to $\rho$. Namely,
$v\cdot P_0=0$, and therefore
\begin{eqnarray*}
v\cdot(P-P_0)=v\cdot P
\end{eqnarray*}
\end{proof}

\begin{claim}
\begin{eqnarray*}
(P-P_0)^t=P^t-P_0
\end{eqnarray*}
\end{claim}

\begin{proof}
Use the fact that $P_0\cdot P=P\cdot P_0=P_0^2=P_0$ and the binomial
expansion.
\end{proof}

\begin{claim}
\begin{eqnarray*}
\lim_{t\to\infty}M^t=0
\end{eqnarray*}
\end{claim}

\begin{proof}
Consider the Jordan decomposition $\Gamma$ of $M$.

\begin{eqnarray*}
\Gamma^t=\left[
\begin{array}{cccc}
B_1^t & 0 & \cdots & 0\\
0 & B_2^t & \cdots & 0\\
\ldots & \ldots & \ddots & \vdots\\
0 & 0 & \cdots & B_n^t
\end{array}
\right]
\end{eqnarray*}

For each Jordan block $B_i$, the corresponding eigenvalue
$\lambda_i$, has absolute value less than $1$. Let $B_i=\lambda_i
I+F$ is an $m\times m$ matrix, then
\begin{eqnarray*}
B_i^t&=&(\lambda_i I+F)^t\\
&=&\lambda_i^{t-m+1}\left(\lambda_i^{m-1}I+{t\choose
1}\lambda_i^{m-2}F+\cdots+{t\choose m-1}\lambda_i^0F^{m-1}\right)
\end{eqnarray*}
One may observe that $B_i^t\to 0$ as $t\to\infty$.
\end{proof}

Combination of these claims gives the convergence of $P$ to $P_0$,
namely, the following theorem.
\begin{theorem}
The random walk on a directed graph $G$ converges to a unique
stationary distribution if $G$ is strongly connected and is not
periodic.
\end{theorem}
