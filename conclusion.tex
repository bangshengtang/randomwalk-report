At current stage of this survey, we just write down what we learned
- some basic knowledge. The techniques are not quite hard, and those
theorems are just an prelude of many more deep theorems.

But the most important thing, we think, is to have an outline of the
framework. There are three pillars in this line of research: the
stochastic one (random walk and Markov chain), the algebraic one
(spectral method) and the geometric one (expansion and expander
graph). The insight that these three different points of view relate
intimately with each other is quite deep, and some perspective helps
to solve many questions that is hard to answer from another
perspective. So it is quite beneficial to study this area.

We still have many things left to be done. From the mathematical
aspect, we have't covered some of the important notions of random
walk: the hitting times, covering times and so on. Also the method
of electrical network seems to be interesting. There are more things
to read from the application side. Of course we need some time to
see how the construction of expanders help to solve the
st-connectivity problem, but some applications, like hardness
amplification should be more reachable.
