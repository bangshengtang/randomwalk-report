\subsection{Basic Notations}
Let $[n]$ denote the set of integers ${1, 2, \dots, n}$.

Remember a finite graph $G$ is a two tuple $(V, E)$. So
$V=[n]$ is the set of vertices, and an edge is represented as the two tuple $(i, j)$, $i, j\in[n]$.
So $E$ can be regarded as a relation $E: V\times V \rightarrow \mathbb{N}$, such that $E(i, j)$
represents the number of edges between $i$ and $j$. Or if we do not allow multiple edges, $E$
can be thought of as some subset of $2^{V\times V}$, such that if there is an edge between
$i$ and $j$ if and only if $(i, j)\in E$. Most of the time we would be dealing with
graphs that don't have multiple edges.

The graph is called undirected if $(i, j)\in E$ implies $(j, i)\in E$. The graph
is directed if the above condition doesn't hold. The neighbor of a vertex $i$ is $\{k|k\in V,
(i, k)\in E\}$, denoted as $Nb(i)$. The adjacency matrix $A_G$ of a graph $G$ is just
$A(i, j)=E(i, j)$.

A (discrete finite) Markov chain is stochastic process $\{X_n, n\geq 0|X_n\in [m]\}$ such that
for any $i_0, i_1, \dots, i_n, \\ i_{n+1}\in S$, $n\in \mathbb{N}$, and $\Pr [X_j=i_j, j\in [n]]>0]$,
we have

\begin{eqnarray*}
\Pr [X_{n+1}=i_{n+1}|X_j=i_j, j\in [n]]=\Pr [X_{n+1}=i_{n+1}|X_n=i_n]
\end{eqnarray*}

The intuition behind the Markov chain is that the future only depends on the current
state, while having no relation with the history. Then one can define the (one step) transition
matrix $P(i, j)$ of a Markov chain as the probability that $i$ transfers to $j$ in one step.
And we can model the state at time $t$ as the distribution vector $\alpha_t$, with $\alpha_t(i)$
to be the probability that the system is at state $i$ at time $t$. So $\alpha_{t+1}=\alpha_t P$.

\subsection{Random Walk on Graphs}

Having defined the basic notions of graphs and Markov chain, let's look at what's the
random walk on graph like. Let the state space be the set of vertices $V$. When
the walker is at a specific vertex $i$, it flips a possibly biased coin to
select one vertex randomly from its neighbors and
goes there. Often we would have the coin unbiased, that is the probability of going
to any neighbor is the same. It is clear that random walk on graphs is a Markov chain.

For a graph $G=(V,E)$ we define the \emph{transition probability matrix} $P_G(u,v)$ by
\begin{eqnarray*}
P(u,v)=\left\{
\begin{array}{cc}
1\over d_u & \textrm{if u $\sim$ v}\\
0 & \textrm{otherwise}
\end{array}
\right.
\end{eqnarray*}
where $d_u$ denotes the degree of a vertex $u$ and $u\sim v$ notation for
$\{u,v\}\in E$.

We represent a distribution on $G$ by a row vector $f$ whose $i$th
entry gives the ``amount'' residing at the $i$th vertex $v_i$. A
typical initial distribution might be
$\overrightarrow{\mathbf{1}}={1\over |V|}(1,1,\dots,1)$, or the row
vector $e^*_i$ with all entries zero except for a one in the $i$th
entry.

If $f$ is the initial distribution, then $fP$ is the distribution after one step,
and $fP^t$ is the distribution after $t$ steps. In a sense, we are considering all
possible random walks simultaneously; for instance, if $f=e^*_i$, the $j$th entry of
$fP^t$ is the likelihood that if we start a walk at vertex $v_i$ and continue for
$t$ random steps that we will find ourselves at vertex $v_j$.

\subsection{Stable Distributions}
A distribution $\pi$ is said to be \emph{stable} if the following condition is
satisfied,
\begin{eqnarray*}
\pi = \pi \cdot P
\end{eqnarray*}

For a regular graph, it is easy to check that, the uniform distribution
$\overrightarrow{\mathbf{1}}$ is a stable distribution.

Generally, if $G=(V,E)$ is an undirected graph and $S\subseteq V$ is arbitrary, then
define the \emph{Volume} of $S$ by
\begin{eqnarray*}
\text{vol} S&=&\sum_{v\in S}d_v\\
&=&\sum_{v\in S}\text{vol} v
\end{eqnarray*}
vol$G$ is defined to be vol$V$.

For an undirected graph $G=(V,E)$, one may check that the following distribution
$\pi$
\begin{eqnarray*}
\pi&=&\left(d_u\over\text{vol}G\right)\\
&=&\left({d_1\over \text{vol}G}, \dots, {d_n\over\text{vol}G}\right)
\end{eqnarray*}
is stable.

Similarly, if $G$ is directed, then let $d^+_u$ and $d^-_u$ denote the outdegree and
indegree of the vertex $u$, respectively. Here we do not define the volume of an
arbitrary subset of $V$; we define only
\begin{eqnarray*}
\text{vol}G=\sum_{u\in V}d^+_u=\sum_{v\in V}d^-_v
\end{eqnarray*}
But we are not so lucky when considering stable distribution. Define $\pi^+$ and
$\pi^-$
\begin{eqnarray*}
\pi^+=\left({d^+_u\over\text{vol}G}\right)\\
\pi^-=\left({d^-_u\over\text{vol}G}\right)
\end{eqnarray*}
Then $\pi^+P=\pi^-$, so that there is no hope of obtaining a stable distribution as
we have done above for undirected graphs. In fact, the equation $fP=f$ has no closed
form solution for $f$ in general.
