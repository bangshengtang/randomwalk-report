\subsection{Four Matrices Related to a Graph}
For some intuition and a survey of spectral graph theory, see
\cite{spectral}. 

For a $n$-vertices graph $G=(V, E)$, the \emph{adjacency matrix} $A$
for this graph is defined as:

\begin{eqnarray*}
A(u,v)=\left\{
\begin{array}{cc}
1 & \textrm{if u $\sim$ v}\\
0 & \textrm{otherwise}
\end{array}
\right.
\end{eqnarray*}

And the \emph{transition matrix} $P$ is simply the adjacency matrix
normalized so that it is a stochastic matrix. For some basic facts
of the spectrum of $P$, please refer to the last section.

\begin{eqnarray*}
P(u,v)=\left\{
\begin{array}{cc}
1\over d_u & \textrm{if u $\sim$ v}\\
0 & \textrm{otherwise}
\end{array}
\right.
\end{eqnarray*}

To facilitate further study, we define another matrix related to
$G$: the (combinatorial) Laplacian matrix $L$. The definition is
simple: the combinatorial Laplacian matrix $L$ is just $D-A$.

\begin{eqnarray*}
L(u,v)=\left\{
\begin{array}{cc}
-1 & \textrm{if u $\sim$ v}\\
1 & \textrm{if $u=v$} \\
0 & \textrm{otherwise}
\end{array}
\right.
\end{eqnarray*}

The quadratic form of the combinatorial Laplacian has a beautiful
expression, and is sometimes referred to as the \emph{Dirichlet
sum}. Notice that for the $i$th node,
$(fL)(i)=(f(D-A))(i)=f(i)d_i-\sum_{x\sim i}f(x)=\sum_{x\sim
i}(f(i)-f(x))$, so

\begin{eqnarray*}
fLf^*=\sum_i \sum_{x\sim i}f(i)(f(i)-f(x))=\sum_{i\sim
x}(f(x)-f(y))^2
\end{eqnarray*}

It is immediately seen that the Laplacian is positive semi-definite,
so its eigenvalues are non-negative.

Another matrix of great importance is the normalized Laplacian
$\mathcal{L}$, the relation of which with combinatorial Laplacian is
similar with the relation of transition matrix $P$ with adjacency
matrix $A$. Recall that $M=D^{-\frac{1}{2}}PD^{-\frac{1}{2}}$, then
Normalized Laplacian $\mathcal{L}$ is defined as
$I-M=I-D^\frac{1}{2}PD^{-\frac{1}{2}}=I-D^{-\frac{1}{2}}AD^{-\frac{1}{2}}$.
From the definition we know $\mathcal{L}=D^\frac{1}{2} L
D^\frac{1}{2}$.

\begin{eqnarray*}
\mathcal{L}(u,v)=\left\{
\begin{array}{cc}
-1\over \sqrt{d_ud_v} & \textrm{if u $\sim$ v}\\
1 & \textrm{if $u=v$} \\
0 & \textrm{otherwise}
\end{array}
\right.
\end{eqnarray*}

Since $M$ is similar to $P$, so the eigenvalues of $\mathcal{L}$ is
just $\{1-\lambda_i\}_{i\in[n]}$. Let $\mu_i=1-\lambda_i$, so
$\mu_i$ are ascending and $\mu_1=0$, and $\mu_n\leq2$. Let
$\{f_i\}_{i\in[n]}$ be the set of eigenvectors of $\mathcal{L}$.


\subsection{Basic Properties of Laplacian matrices}

Now we turn our focus to the eigenvalues of $\mathcal{L}$. Here are
some simple facts, the proofs of which are just a few lines. The
interesting thing here is that, by considering Laplacian, we get
another proofs of the properties of spectrum of transition matrix
$P$. Furthermore, using Laplacian, especially the very useful
quadratic form of combinatorial Laplacian, we can easily answer the
questions proposed in the last section.

In order to use the quadratic form of $L$ in the study of
$\mathcal{L}$, some more tools are needed, namely the Rayleigh
quotient and Courant-Fischer characterization of eigenvalues.

\begin{definition}
The \emph{Rayleigh quotient} of a matrix $A$ for a vector $g$,
$R(g)$ is defined as $\frac{gAg^*}{gg^*}$.
\end{definition}

\begin{fact}
Let $f=gD{-\frac{1}{2}}$, then the Rayleigh quotient of Laplacian
$\mathcal{L}$ is
\begin{equation*}
R(g)=R'(f)=\frac{g\mathcal{L}g^*}{gg^*}=\frac{fLf^*}{ff^*}=\frac{\sum_{i\sim
j} (f(i)-f(j))^2}{\sum_i f^2(i) d_i}
\end{equation*}

From now on, when we refer to Rayleigh quotient $R(f)$, we use the
variation $R'(f)=\frac{fLf^*}{ff^*}$. Furthermore, let
$\{f'_i=fD^{-\frac{1}{2}}\}_{i\in[n]}$ denote the variational
eigenvectors, which will be called harmonic eigenvectors below.
\end{fact}

Since $L$ has 0-eigenvector $\overrightarrow{1}=\frac{1}{n}(1, 1,
\dots, 1)$, from the above we know $\mathcal{L}$ has 0-eigenvector
$\overrightarrow{1}D^{\frac{1}{2}}$. Courant-Fischer lemma gives us
a way to characterize the eigenvalues using Rayleigh quotient:

\begin{proposition}
For graph $G$ and let $R(f)$ be its Rayleigh quotient, then the
eigenvalues of $\mathcal{L}$ have the following characterization:

\begin{eqnarray*}
\mu_1 & = &  \overrightarrow{1}D^{\frac{1}{2}}\\
\mu_2 & = &  \inf_{f: \sum_x f(x)d_x=0}R(f)\\
&\dots& \\
\mu_i & = & \sup_{h_k, k\in[i]} \inf_{f: f \vert h_k D, k\in[i]} R(f)\\
&\dots& \\
\mu_n & = & \sup_f R(f)\\
\end{eqnarray*}
\end{proposition}

Now we can use these tools to exploit the spectrum of $\mathcal{L}$.

\begin{fact}
For graphs with no self loops and no isolated vertices, we have
$\sum_{i=0}^{n-1} \mu_i=n.$
\end{fact}

\begin{proof}
Note that in this case $\text{trace}\mathcal{L}=n$ since every
number along the diagonal is 1, and we know $\sum_{i\in[n]}
mu_i=\text{trace}\mathcal{L}. $
\end{proof}

\begin{fact} \label{connected}
The graph $G$ is connected $\Leftrightarrow$ $\mu_2>0$. Moreover,
the graph has $i+1$ components $\Leftrightarrow$ $\mu_i=0$ and
$\mu_{i+1}>0$.
\end{fact}

\begin{proof}
($\Rightarrow$) From the Courant-Fischer description of $\mu_1=0$,
we see that for harmonic eigenvectors, if $i\sim j$ then
$f(i)=f(j)$. So if $G$ is connected, then 0-eigenvectors need to
have $f(i)=f(j)$ for any $(i, j)\in E$, thus its eigenspace has
dimension 1, and $\mu_2>0$.

($\Leftarrow$) By contradiction, if $G$ is not connected, let the
two connected components be $G_1$ and $G_2$. Then the spectrum of
$G$ is the union of $G_1$ and $G_2$ so eigenvalue 0 appear at least
twice in $G$'s spectrum.
\end{proof}

\begin{fact} \label{bipartite}
The graph is bipartite $\Leftrightarrow$ $\mu_n=2$.
\end{fact}

\begin{proof}
($\Rightarrow$)

\begin{eqnarray*}
R(f) & = & \frac{\sum_{i\sim
j} (f(i)-f(j))^2}{\sum_i f^2(i) d_i} \\
& \leq & {{\sum_{i\sim j} (f^2(i)+f^2(j))}\over{\sum_i (f(i))^2 d_i}} \\
& = & \frac{2\sum_i f^2(i) d_i}{\sum_i f^2(i) d_i} \\
& = & 2
\end{eqnarray*}

So if the graph is bipartite, let $V=V_1 \cap V_2$ and $V_1$ and
$V_2$ are disjoint. And we construct $f$ so that $f(i)=1$ if $i\in
V_1$ and $f(i)=2$ if $i\in V_2$. Then from the above deduction and
using the Courant-Fischer lemma of the largest eigenvalue, the
result holds.

($\Leftarrow$) Since $f'_n$ is the harmonic eigenvector associated
with the eigenvalue 2, and it goes through the equality of the above
deduction, we know $f(i)=-f(j)$ for $x\sim y$. So $G$ is
two-colorable and thus bipartite.
\end{proof}

So Fact (\ref{connected}) and Fact (\ref{bipartite}) give an answer
to the question for which type of graphs the random walk converges.
It turns out that as long as the graph is connected and
non-bipartite, the random walk always converges to its unique
distribution, namely $\pi=(\frac{d_u}{\text{vol}G})$.

\subsection{Explanation from the Perspective of Markov Chain}
Since random walk on graph is a Markov process, we would like to
explain the above facts from the perspective of stochastic process.
From the textbook we know the two basic facts:

\begin{enumerate}
\item Irreducible, aperiodic and finite Markov chain has a unique
stationary distribution;
\item Irreducible, aperiodic Markov chain is positive recurrent if
and only if it has a stationary distribution. At this time, its
stationary distribution is just its limit distribution.
\end{enumerate}

From the perspective of the two theorems, we only need to find the
conditions of a random walk on graph is irreducible and aperiodic.
By simple observation the following claims can be made for
undirected graphs:

\begin{enumerate}
\item A graph is irreducible if and only if it is connected;
\item A graph is aperiodic if and only if it is non-bipartite.
\end{enumerate}

So here is another perspective on the conditions of a graph to be
random walk convergent. But for practical applications, one often
need to build a graph on which the random walk converges
\emph{fast}. From the last section we know the speed of convergence
depends on the \emph{spectral gap} $\alpha_G$ of a graph, so the
problem turns to building graphs with large spectral gap. Before
addressing this problem, another important connection needs to be
made: the expansion of a graph and the spectral gap.
