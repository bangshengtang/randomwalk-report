Random walk on graph is an interdisciplinary area of graph theory
and random walk theory. Due to the importance of graphs in computer
science as a modeling tool, random walk on graphs has become a
frequently-used method in algorithm design and complexity theory.

Some examples of application of random walk are as follows:

\begin{enumerate}
\item In algorithm design, since random walk is simple to implement,
and it has the local property (doesn't depend on the whole structure
of the graph), thus having good adaptability when graph changes, so
it enjoys wide application in communication network;
\item In graph theory, expander is a type of sparse graph that has a
good connectivity. In recent years, the theory of expander has made
a large progress, including its construction, analysis and
application. Since it combines two seemingly contradictory
properties, we see many application in coding theory and
pseudo-randomness theory. An important perspective of expander is
just the random walk: random walk on expanders converges fast (in
time $O(\log n)$;
\item In complexity theory, an important result is that we only need
deterministic log-space to test the connectivity of two vertices.
The basic technique for this result is: first we start with an
arbitrary graph, and consider the random walk on it. By constructing
good expander from the initial graph while preserving the
connectivity, one can show that the connectivity of two nodes is in
log-space;
\item In quantum computing, an important technique of algorithm
design is the quantum counterpart of random walk. For some problems,
this technique has helped to design some of the most efficient
algorithms, and some of them are proved to optimal.
\end{enumerate}

In this survey, we are going to study some of the basic concepts and
techniques motivated by random walk on graphs. Our main focus is on
undirected graph, with some efforts on directed case. Another
feature of this survey would be the perspective of the theory of
Markov chains, especially the limit distribution and mixing time.
