\subsection{Introduction} Now we have modeled random walk on graphs
as a Markov process, for which the transition matrix $P$ is the
normalized adjacency matrix. That is, $P(i, j)=\frac{1}{d_i} A(i,
j)$, or let $D$ be the diagonal matrix of degrees (the diagonal
matrix of $n$ by $n$, and $D(i, i)=$the degree of the vertex $i$),
then $P=D^{-1}A$. So for this model, what are the interesting
questions?

From the mathematical perspective, one can definitely ask questions
about \emph{convergence}, that is the behavior of random walk when
the number of steps goes to infinity. Related questions are: whether
the random walk converges, and if so, how fast will it converge?

From the practical perspective, it turns out that convergence
problems are also very important. That's in many applications, random walk is used as
a robust way to explore a graph. Since the specific node we would like to visit is not known in
advance, it would be good for us to know the probability distribution converges
to some distribution that we can calculate, so that the
effect of random walk can be estimated.

In order to give satisfying answers to these questions, we will
first study the spectrums of several matrices related to the
transition matrices. We think this point of view is both interesting
and deep: it turns out a large number of graph properties are intimately
connected with the spectrum of those matrices. In this survey we just cover
two properties: the convergence of random walk and the expansion of a graph.

In this section, we first show how
the spectrum of convergence of random walk is connected with the spectrum of
transition matrix $P$. This motivates us to define some other matrices in the next section,
which, together with $P$, can be viewed as the variations of adjacency matrix $A$.

\subsection{Basic Properties of Transition Matrix}

Our purpose is to investigate how the spectrum of $P$ controls the
convergence of random walk. In our discussion the graph has no
isolated vertices, that is the vertices with degree $=0$.

Let $M=D^{-\frac{1}{2}}PD^{-\frac{1}{2}}$, then $M$ is symmetric and
from linear algebra we know that $M$ has $n$ real eigenvalues and
the dimension of its eigenspace is $n$. Then note that
$P=D^{-1}A=D^{-\frac{1}{2}}MD^\frac{1}{2}$, so $P$ also has the same
property.

Formally, let $\{\lambda_i\}_{i\in[n]}$ be the set of eigenvalues,
with $\lambda_i\geq \lambda_j$ if $i<j$, and let ${e_i}_{i\in[n]}$
denote the corresponding eigenvectors. We just list some basic facts
of spectrum of $P$ without proof here.

\begin{enumerate}
\item $P$ has an eigenvalue $1$ with eigenvector $\pi=(\frac
{d_u}{\text{vol}G})$. (Furthermore, one can see that if the number
of connected components is $k$, then the eigenspace of eigenvalue 1
has dimension $k$);
\item Since $P$ is a stochastic matrix, the absolute values of its
eigenvalues are less or equal to 1. (First notice the fact that the
eigenvalues of $P^T$ and $P$ are the same. Then for some eigenvalue
$\lambda$ and its eigenvector $(x_i)_{i\in [n]}$, and let $x_k$ be
the component with largest absolute value. So one can easily see
that $|\lambda x_i|=|\sum_j P^T_{ji} x_j|\leq \sum_j |P^T_{ji} x_j|
\leq x_i$ so $|\lambda|\leq 1$.)
\end{enumerate}

\subsection{Transition Matrix and Convergence}

The intuition that the spectrum of $P$ controls the convergence of
random walk is as follows: from the Markov property of random walk,
we know that given the initial distribution $v$ and the transition
matrix $P$, the probability distribution after $t$ steps is just
$vP^t$. Since $P$ is symmetric, we know $P$, as a linear operator,
has a set of real eigenvalues and a set of corresponding
eigenvectors which are orthogonal. That is, if we view the initial
distribution decomposed in the eigenbasis, the action of $P$ is just
multiplying the eigenvalues along the components in eigenbasis. So
if the absolute values of the eigenvalues but one are less than 1,
after many steps the distribution vector is just the dominating
eigenvector.

Formally, we have the following lemma answering the question:
whether the random walk will converge or not.

\begin{lemma}
Given an undirected graph $G=(V, E)$ without isolated vertices, let
$P$ be its transition matrix, and
$M=D^{-\frac{1}{2}}PD^{-\frac{1}{2}}$. $\{\lambda_i\}_{i\in[n]}$ is
the set of eigenvalues of $M$, with $\lambda_i\geq \lambda_j$ if
$i<j$, and $\{e_i\}_{i\in[n]}$ denotes the corresponding
eigenvectors, with $\parallel e_i\parallel =1$. Let
$\alpha_G=\max{(|\lambda_2|, |\lambda_n|)}$. If $\alpha<1$ then the
random walk on $G$ will converge to $\pi=(\frac {d_u}{\text{vol}G})$
for any initial distribution $f$.
\end{lemma}

\begin{proof}
Since $e_i$ form an orthonormal basis, the effect of $M$ as a linear
operator can be decomposed as $\sum_{i\in [n]} \lambda_i e_i^* e_i$,
in which $e_i^* e_i$ can be understood as the projection onto the
direction of $e_i$. Also note that $\lambda_1=1$ and
$e_1=\frac{\sqrt{d_u}}{\sqrt{\text{vol}G}}$. Then

\begin{align}
\parallel fP^t-\pi \parallel &= \parallel
fD^{-\frac{1}{2}}M^tD^{\frac{1}{2}}- \pi \parallel \\
&=\parallel fD^{-\frac{1}{2}}\sum_{i\in[n]\setminus
\{1\}}(\lambda_i^t e_i^*
e_i)D^{\frac{1}{2}}+fD^{-\frac{1}{2}}\lambda_1 e_1^* e_1
D^{\frac{1}{2}}- \pi \parallel \\
&=\parallel fD^{-\frac{1}{2}}\sum_{i\in[n]\setminus
\{1\}}(\lambda_i^t e_i^* e_i)D^{\frac{1}{2}} \parallel \\
&\leq \max_{i\in[n]\setminus \{1\}}|\lambda_i|^t\cdot
(\frac{\max_u\sqrt{d_u}}{\min_v{\sqrt{d_v}}}) \\
&= \alpha^t \cdot(\frac{\max_u\sqrt{d_u}}{\min_v{\sqrt{d_v}}})
\end{align}

So if $\alpha<1$, this inequality gives the existence and uniqueness
of a stationary distribution.
\end{proof}

Furthermore, this inequality will give us an idea of how \emph{fast}
the random walk will converge: it depends on the \emph{eigenvalue
gap} $1-\alpha$.

\begin{definition}
If the random walk on $G$ converges, then the mixing time $t$ of a
random walk on graph $G$ is $t=\inf_y{\parallel yP^t-\pi \parallel
\leq \frac{1}{n^3}}$.
\end{definition}

\begin{corollary}
If for graph $G$, $\alpha_G < 1$, then the mixing time is of order
$O(\log n/(1-\alpha_G))$.
\end{corollary}

\begin{proof}
Since $\parallel fP^t-\pi \parallel \leq \alpha^t
\cdot(\frac{\max_u\sqrt{d_u}}{\min_v{\sqrt{d_v}}})$ for any $f$, so
if we require $\parallel fP^t-\pi \parallel \leq \frac{1}{n^3}$ we
just need $\alpha^t
\cdot(\frac{\max_u\sqrt{d_u}}{\min_v{\sqrt{d_v}}}) = \frac{1}{n^3}$.
That is $t=\frac{\log n} {\log (1/\alpha)}$. Since most of the time
$\alpha$ is close to 1, there is an approximation $\log (1/\alpha) =
1-\alpha$ and the result follows.
\end{proof}

So not only $\alpha$ reflects the convergence of the random walk,
but also the spectral gap $1-\alpha_G$ controls the speed of the
convergence. Now the question goes to determine which type of graphs
has large eigenvalue gap, and in order to answer this question, we
first study some other matrices that are related to a graph, namely
the Laplacian matrix of a graph.
