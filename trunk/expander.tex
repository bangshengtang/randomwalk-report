In order to introduce expander graph, the following definitions are of great importance.

\subsubsection{Cheeger's Constant}
Suppose we are given a graph $G=(V,E)$ and a subset $S\subseteq V$. We wish to define the following
two sets:
\begin{eqnarray*}
	\partial S=\{\{u,v\}|u\in S, v\not\in S\}
\end{eqnarray*}
and
\begin{eqnarray*}
	\delta S=\{v\not\in S|v\sim u, u\in S\}
\end{eqnarray*}

\begin{definition}
	The \emph{Cheeger Ratio} for a vertex set $S$ is
	\begin{eqnarray*}
		h(S)={|\partial S|\over\min{\text{vol}(S),\text{vol}(\bar{S})}}
	\end{eqnarray*}
	where, $\bar{S}=V-S$
\end{definition}

\begin{definition}
	For any graph $G=(V,E)$, the \emph{Cheeger Constant} of $G$ is given by
	\begin{eqnarray*}
		h_G=\min_{S}h(S)
	\end{eqnarray*}
\end{definition}

\subsubsection{Cheeger's Inequality}
\begin{theorem}
	For any graph $G$,
	\begin{eqnarray*}
		2h_G\ge \lambda_1\ge {h_G^2\over 2}
	\end{eqnarray*}
\end{theorem}

\subsubsection{Expander Graph}


\subsubsection{Facts}
